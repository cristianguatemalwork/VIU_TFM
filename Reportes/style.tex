%\usepackage[utf8]{inputenc}
\usepackage[spanish]{babel}
\usepackage{amsmath}
\usepackage{amsfonts}
\usepackage{amssymb}
\usepackage{graphicx}
\usepackage[left=2.75cm,right=2.6cm,top=3.5cm,bottom=3.5cm]{geometry}



\usepackage{titletoc}
% Keywords command

\providecommand{\keywords}[1]
{
  \small	
  \textbf{\textit{Palabras clave:\hspace{0.3cm}}} #1
}

\usepackage{draftwatermark}

\usepackage{colortbl}
\usepackage{xcolor}
%\definecolor{naranja}{RGB}{230, 81, 19}
\definecolor{naranja}{HTML}{E65113}
\usepackage[shortlabels]{enumitem}
%\definecolor{slcolor}{RGB}{230, 81, 19}
\definecolor{slcolor}{HTML}{E65113}
\newcommand{\headlinecolor}{\color{slcolor}}
\usepackage{titlesec}

\definecolor{gray75}{gray}{0.75}
\newcommand{\hsp}{\hspace{-10pt}}

%%%%%%Comandos adicionales
\usepackage{csquotes}
\usepackage{natbib}
\usepackage{longtable}
\usepackage{multirow}
\usepackage{multicol}
\usepackage{color}
\usepackage{xcolor}
\usepackage{colortbl}

\usepackage{longtable}  
\usepackage{float}      
\usepackage{booktabs} 
\usepackage{listings}

\lstdefinestyle{sqlstyle}{
    language=SQL,
    keywordstyle=\color{blue},
    commentstyle=\color{gray},
    stringstyle=\color{red},
    morekeywords={SELECT, INSERT, UPDATE, DELETE, FROM, WHERE, JOIN}, % Añadir más palabras clave si es necesario
    breaklines=true,
    frame=single,          % Marco alrededor del código
    framesep=5pt,          % Separación entre el marco y el código
    rulecolor=\color{black}, % Color del marco
    %backgroundcolor=\color{black!10}, % Color de fondo (opaco)
    inputencoding=utf8,
    xleftmargin=1.5em,   % Margen a la izquierda
    xrightmargin=1.5em,  % Margen a la derecha
    basicstyle=\setlength{\lineskip}{0pt}\fontsize{5}{6}\selectfont\ttfamily,
    literate={á}{{\'a}}1 {é}{{\'e}}1  {í}{{\'i}}1 {ó}{{\'o}}1 {ú}{{\'u}}1 {ñ}{{\~n}}1 
             {Á}{{\'A}}1 {É}{{\'E}}1  {Í}{{\'I}}1 {Ó}{{\'O}}1 {Ú}{{\'U}}1 {Ñ}{{\~N}}1
             {\$}{{$\$$}}1 {\_}{{$\_$}}1 {\/}{{$/$}}1,
}

\usepackage[colorlinks = true, linkcolor = black, urlcolor = naranja, citecolor = black,
  linktoc = page]{hyperref}
  
\setlength{\parskip}{0.9\baselineskip} % espacio entre parrafos
\usepackage{setspace}


\usepackage{etoolbox}  

\newcommand{\citetitle}[1]{%
    \textit{\bibentry{#1}}%
}
\usepackage{bibentry}     % Para usar \bibentry

%%%%%%%%%%%%%%%%%%%%%%%%%%%%%%%%%%%%%%%%%%%%%%%%%%%%%%%%%%%%%%%%%%%%%%%%%%%%%
%------------------COMANDOS PARA EL TIPO DE LETRA---------------------------%
%\usepackage{fontspec}
\usepackage[T1]{fontenc}
\usepackage{helvet}
\renewcommand{\familydefault}{\sfdefault}

\titleformat{\chapter}[hang]{\vspace{-3cm}\headlinecolor\Huge\bfseries}{\thechapter.\hsp}{20pt}{\Huge\bfseries}

%\titleformat{\section}[hang]{\Large\bfseries}{}{20pt}{\Large\bfseries}

\titleformat{\subsection}[hang]{\normalsize\bfseries}{}{20pt}{\large\bfseries}
\titleformat{\appendix}[hang]{\vspace{-3cm}\headlinecolor\Huge\bfseries}{\thechapter.\hsp}{20pt}{\Huge\bfseries}


%%%%%%%%%%%%%%%%%%%%%%%%%%%%%%%%%%%%%%%%%%%%%%%%%%%%%%%%%%%%%%%%%%%%%%%%%%%%%
%-------------------COMANDOS PARA TABLAS  E IMAGENES ---------------------%
\usepackage{tikz}
\usepackage{tabularx}

%%%%%%%%%%%%%%%%%%%%%%%%%%%%%%%%%%%%%%%%%%%%%%%%%%%%%%%%%%%%%%%%%%%%%%%%%%%%%
%-----------------COMANDOS PARA CABECERAS Y PIE DE PAGINA -----------------%
\usepackage{lastpage}
\usepackage{fancyhdr}
\usepackage{titlesec}

\fancypagestyle{plain}{%
  \renewcommand{\headrulewidth}{0pt}
\fancyhead{}
\fancyfoot{}
\fancyfoot[R]{{\scriptsize\thepage\ de \pageref{LastPage} | Detección de aportaciones fraudentas de los afilaidos del Instituto Ecuatoriano de Seguridad Social}}
\fancyhead[L]{\tikz[remember picture,overlay]\node[opacity=0.4] at (-3mm, 10mm){\includegraphics[scale=0.18]{graficos/image3.png}};}
\fancyheadoffset{0pt}
}

\pagestyle{plain}

